% LaTeX file for resume 
% This file uses the resume document class (res.cls)

\documentclass{res} 
%\usepackage{helvetica} % uses helvetica postscript font (download helvetica.sty)
%\usepackage{newcent}   % uses new century schoolbook postscript font 
\newsectionwidth{0pt}  % So the text is not indented under section headings
\usepackage{fancyhdr}  % use this package to get a 2 line header
\renewcommand{\headrulewidth}{0pt} % suppress line drawn by default by fancyhdr
\setlength{\headheight}{24pt} % allow room for 2-line header
\setlength{\headsep}{24pt}  % space between header and text
\setlength{\headheight}{24pt} % allow room for 2-line header
\pagestyle{fancy}     % set pagestyle for document
\cfoot{}                                     % the foot is empty
\topmargin=-0.5in % start text higher on the page

\begin{document}
\thispagestyle{empty} % this page has no header  
\name{CAROLYN A. GULAS\\[14pt]}% the \\[12pt] adds a blank line after name

\address{ 5405 Hedge Creek Ln NW \\   Acworth, GA 30101 \\ +1 724 255 8968}      
                                      
\address{cgulas@ona.io}

\begin{resume}
 
\section{PROFESSIONAL EXPERIENCE} 

\vspace{8pt}
\textbf{Ona} \hfill        \textbf{January 2014-Present} \\
\emph{mHealth Specialist}       \hfill   Atlanta, GA

\begin{itemize} \itemsep -2pt % reduce space between items
	\item Project lead on the development and implementation of the Smart Registry platform in multiple country settings

 \end{itemize} \vspace{-4pt}

\vspace{8pt}
\textbf{Foundation for Research in Health Systems / World Health Organization} \hfill        \textbf{July 2012-December 2013} \\
\emph{mHealth Project Consultant}       \hfill   Mysore, Karnataka, India

\begin{itemize} \itemsep -2pt % reduce space between items
	\item Developed appropriate RMNCH content for the Dristhi mobile health platform in line with GoI and NRHM objectives
	\item Liaised regularly with technology partners in India and elsewhere to ensure timely and correct incorporation of content into the application
    \item Developed field test protocols for new features of the Dristhi mobile health platform
	\item Created training materials and oversaw training of end users and implementation of the application in the field
	\item Led the ongoing documentation of Dristhi staff decision-making processes and lessons learned throughout the project
	\item Created and operationalized a research protocol for large randomized controlled trial of Dristhi
 \end{itemize} \vspace{-4pt}

\textbf{mHealth Alliance} \hfill        \textbf{January-June 2012} \\
\emph{Program Intern}       \hfill   Washington, DC
   \begin{itemize} \itemsep -2pt % reduce space between items
	\item Assisted with the management of the UN Secretary-General’s Innovation Working Group catalytic grant mechanism 
	\item Facilitated the convening of the Alliance’s working groups
	\item Created and operationalized the Alliance’s internal monitoring and evaluation framework
 \end{itemize} \vspace{-4pt}

\textbf{Johns Hopkins Center on Aging and Health (COAH)} \hfill \textbf{March 2011-June 2012} \\
\emph{Qualitative Research Consultant} \hfill Baltimore, MD
 \begin{itemize} \itemsep -2pt
  \item  Advised researchers on best strategy for qualitative analysis of the Experience Corps project data and assisted with data analysis using Atlas.ti qualitative software
\end{itemize} \vspace{-6pt}

\textbf{World Health Organization} \hfill \textbf{Summer 2011} \\
\emph{Intern} \hfill Geneva, Switzerland
 \begin{itemize} \itemsep -2pt
  \item Established metrics for comparing mHealth project success
  \item Determined criteria in support of innovation scale-up
  \item Reviewed content and functionality of mHealth solutions
  \item Developed a framework for assessing the performance and impact of mHealth projects focused on maternal and child health
 \end{itemize} \vspace{-6pt}
 
\textbf{Johns Hopkins Center for Immunization Research (CIR)} \hfill \textbf{Spring 2011} \\
\emph{Graduate Student Assistant} \hfill Baltimore, MD
 \begin{itemize} \itemsep -2pt
  \item Contacted and pre-screened health volunteers for phase I vaccine clinical trials for dengue and shigella
\end{itemize} \vspace{-6pt}

\textbf{Department of Anthropology Honors Program, Emory University} \hfill \textbf{August 2009-May 2010} \\
\emph{Honors Student Researcher} \hfill Atlanta, GA
 \begin{itemize} \itemsep -2pt
  \item Created and disseminated online survey and conducted semi-structured interviews among Emory undergraduate students on the topics of HPV vaccine acceptance and sexual risk perceptions
  \item Earned highest honors at oral defense in April 2010.  
\end{itemize} \vspace{-6pt}

\section{EDUCATION} 
\vspace{8pt} 
\textbf{Johns Hopkins Bloomberg School of Public Health}  \hfill \textbf{May 2012} \\
\emph{Master of Science in Public Health (MSPH) in International Health with a Concentration in Social and Behavioral Interventions}, GPA 4.00 \hfill \textbf{Baltimore, MD}

\vspace{8pt} 
\textbf{Emory University}  \hfill \textbf{May 2010}\\
\emph{Bachelor of Science in Anthropology and Human Biology, summa cum laude}, GPA 4.00 \hfill \textbf{Atlanta, GA}

\section{AWARDS \& HONORS}
\begin{itemize} \itemsep -2pt
	\item 2010 Charles Elias Shepard Scholarship (for graduate study)
	\item 2010 Highest Honors, Emory University Department of Anthropology
    \item 2009 Phi Beta Kappa, Emory University Chapter
	\item 2008 Emory College Outstanding Sophomore Student Award
	\item 2008 Phi Eta Sigma, Emory University Chapter
	\item 2007 Emory College Class Gift Book Award
	\item 2006-2010 Emory College Dean’s List
\end{itemize}

\section{PUBLICATIONS}
\begin{itemize} \itemsep -2pt
	\item Gulas, C., Mehl, G., & Labrique, A.  (in process).  INFORM: Implementation and Evaluation Framework for mHealth, a framework for expanding the mHealth evidence base and providing implementation and evaluation guidance for mHealth projects worldwide.
    \item Dalglish, S.L., Pham, H., Wilkinson, R., Gulas, C., Poulsen, M.N., Hulland, K.R.S., Winch, P.J. (in process). Growing plants, growing children: a qualitative study of the nutritional, social, and cognitive benefits of children’s involvement in gardening in Baltimore City.
    \item Poulsen, M., Solawetz, K., & Gulas, C. Growing an Urban Oasis in Baltimore City: Gardeners’ Perceptions of the Benefits of Community Gardening.  Accepted for publication in Culture, Agriculture, Food and Environment.
    \item Gulas, C.  (2010).  Acetic Acid Innovation.  Lambda Alpha Journal (National Collegiate Honors Society for Anthropology).
    \item Gulas, C.  (2010).  Knowledge and acceptance of the HPV vaccine and sexual risk-taking, a study of undergraduate students at Emory University.  Lambert Academic Publishing. 
    \item Hadley, C., Patil, C. L., & Gulas, C. (2010). Social Learning and Infant and Young Child Feeding Practices. Current Anthropology, 51(4), 551-560. doi: 10.1086/653998
    \item October 2009: Hadley, C., Weaver, J., Englander, L., and C. Gulas. Translating Media Reports Into a Global Food Security Monitoring Tool. Society for Medical Anthropology Annual Meeting. New Haven, CT.
\end{itemize}

\section{UNIVERSITY AND PROFESSIONAL SERVICE}
\begin{itemize} \itemsep -2pt
	\item Emory College Office of Student Conduct, Appeal Board Appointed Justice (September 2009-May 2010)
    \item Emory Undergraduate Research Journal (EURJ), Editor in Chief (September 2007-May 2010)
    \item Emory College Office of Student Conduct, Peer Review Board Appointed Justice (September 2007-May 2009)
\end{itemize}

\section{LANGUAGES} 
\vspace{8pt}
English (Native), French (Conversational), Kannada (Basic)

\section{GEOGRAPHIC WORK EXPERIENCE} 
\vspace{8pt}
Tanzania, India

\section{ANALYTIC SOFTWARE} 
\vspace{8pt}
Qualitative:  Atlas.ti, NVivo, Visual Anthropac;
Quantitative:  Stata, SPSS, Microsoft Excel

\end{resume} 
\end{document}













